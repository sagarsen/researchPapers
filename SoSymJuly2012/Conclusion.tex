\section{Conclusion}
\label{sec:Conclusion}

Testing model transformations  presents the challenging problem of developing approaches to automatically generate effective test models. In this paper we present {\Pramana}, a tool to generate models conforming in the input domain of a transformation and guided by different coverage criteria.  First, {\Pramana} helps us precisely specify the input domain of a model transformation incremental pre-condition improvement.  Second, we use {\Pramana} to generate sets of test models that compare coverage and unguided strategies for model generation. All test sets using these strategies detect faults given by their mutation scores.  The comparison of coverage strategies with unguided generation taught us that both strategies {\AllPartitions} and {\AllRanges} look very promising. Coverage strategies give a maximum mutation score of 93\% compared to a maximum mutation score of 70\% in the case of unguided test sets. We observe that mutation scores do not vary drastically despite the generation of multiple solutions for the same test strategy.  We conclude from our experiments that the {\AllPartitions} strategy is a promising strategy to consistently generate a small test of test models with a good mutation score. However, to improve effectiveness of test sets we might require effort from the test designer to obtain test model knowledge/test strategy that take the internal model transformation design requirements into account. The experiments in this paper were performed based on the mutation analysis of {\transfo} written in the language {\Kermeta}. In future, we intend to develop mutation analysis tools for various mature model transformation languages. The automatic mutation analysis tool will help us perform experiments using a number of transformation case studies. Applying our approach to large input metamodels such as the {\UML} is a challenge in scaling our approach. We intend to leverage our recently developed technique called \emph{metamodel pruning} \cite{sen2009b} to extract a small subset of large metamodel such as {\UML} which is conducive to constraint solving in {\Pramana} and consequently model generation for large input metamodels. 
